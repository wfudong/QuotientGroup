\documentclass{article}
\usepackage{amsmath,amsthm,amssymb}
\usepackage{color}
\usepackage{hyperref}
\usepackage{indentfirst}% first line indent
\usepackage{tikz}% a draw tool
\usepackage[utf8]{inputenc}% hat over letter
\theoremstyle{plain}
\newtheorem{thm}{Theorem}[section]

\theoremstyle{definition}
\newtheorem{defn}{Definition}[section] % definition numbers are dependent on theorem numbers
\newtheorem{ex}{Example}[section] % same for example numbers

\newcommand{\normal}{\vartriangleleft}% symbol of normality

\title{\textbf{Normality, Quotient Groups, And Homomorphism}}
\author{Fudong}
\begin{document}
\maketitle
\section{Introduction}
Normal subgroups play a remarkablily important role in determining both the structure of a group $G$ and the homomorphisms with domain $G$. This note derives from \href{http://www.amazon.com/Algebra-Graduate-Texts-Mathematics-v/dp/0387905189}{\textcolor{blue}{\textbf{Hungerford's Algebra}}} textbook Chapter I section 5. The notations inherit this book.
\section{Definition}

\begin{defn}[\textbf{normal subgroup}]
A subgroup $N$ of a group $G$ satisfies $aN=Na,\forall a \in G$ is said to be \em{normal} in $G$. write as $N\normal G$.
\end{defn}
\begin{defn}[\textbf{quotient group or factor group}]
$G/N$ is the set of all cosets of $N$ in $G$. $N\normal G$, we call $G/N$ \em{quotient group} of $G$ by $N$. 
\end{defn}
\begin{defn}[\textbf{canonical epimorphism or projection}]
The map $\pi:G\rightarrow G/N$ is called \em{canonical epiphism or projection}.
\end{defn}



\section{Themorem and Corollary}
\begin{thm}[\textbf{Quotient group theorem}]
$f:G\rightarrow H$is homomorphsim. $N\normal G$,$N\subseteq \ker f$ $\Rightarrow$ there is a unique homomophism $\overline{f}:G/N\rightarrow H$ s.t. $\overline{f}(aN)=f(a)$ for all $a\in G$.$Im\,f=Im\,\overline{f}$ and $Ker\,\overline{f}=(Ker\,f)/N$.$\overline{f}$ is isomorphism iff f is epimorphism and $N=\ker f$.\\
%\end{thm}
\begin{tikzpicture}
\draw [->] (0,0)--(0,-1);
\node [above] at (0,0) {$G$} ;
\node [below] at (0,-1) {$G/N$};
\draw [->] (0.5,-1)--(4,-0);
\draw [->] (0.2,0.2)--(4+0,0.2);
\node [right,above] at (4.2,-0.1){$H$};
\node [above] at (2,0.2){$f$};
\node [right,below] at (2.3,-0.5) {$\overline{f}$};
\end{tikzpicture}
\end{thm}
\begin{thm}[\textbf{First Isomorphism Theorem}]
If $f:G\rightarrow H$ is a homomorphism, then f induces an isomorphism $G/Ker\, f\cong Im\, f$.
\end{thm}
\begin{thm}[\textbf{Second Isomorphism Theorem}]

\end{thm}
\begin{thm}[\textbf{Third Isomorphism Theorem}]
\end{thm}
\section{Summary and Exercise}

\end{document}